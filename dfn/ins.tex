% Insurance Terminology
%
% Intended to be included within its own section.
%
% Also requires gls-ins-calif definition.

\begindeptgroup{uw}

\sigauth{NL}
For a glossary of insurance terms from the state of California,
see~\cite{gls-ins-calif}.

\begin{description}
  \dt{all other perlis (AOP) deductible}
  Synonymous with property deductible.

  \sigauth{NL}
  \dt{applicant}
  The intended policy owner.

  \dt{cause of loss}
  Basic, special, or broad.

  \sigauth{NL}
  \dt{DEFS}
  Direct-Applied Exterior Finish System; a type of building siding (compare to
  ^[EIFS]).

  \sigauth{NL}
  \dt{EIFS}
  Exterior Insulation and Finish System; a type of building siding (compare to
  ^[DEFS]).

  \sigauth{NL}
  \dt{indication}
  An approximation of the cost of insurance based on incomplete or limited
  information provided by the applicant.

  \sigauth{NL}
  \dt{LRO}
  Lessor's Risk Only---a lessor's risk policy.

  \sigauth{NL}
  \dt{monoline}
  A single line of business (e.g. ^[Property] or ^[General Liability]); contrast
  to ^[package].

  \sigauth{NL}
  \dt{NOC}
  No Other Class Code---catch-all when no other class code is a fit for the
  particular risk.

  \sigauth{NL}
  \dt{package}
  Multiple lines of business; contrast to ^[monoline].

  \sigauth{NL}
  \dt{policy}
  Written insurance contract.

  \sigauth{NL}
  \dt{quote}
  An estimate of the cost of insurance based upon information supplied to the
  insurance company by the applicant.

  \sigauth{NL}
  \dt{risk}
  The individual or property to which the insurance policy or quote relates.

  \sigauth{NL}
  \dt{TRIA}
  Terrorism Risk Insurance Act---signed into United~States federal ^law by
  ^[George W.~Bush] on November~26, 2002.
\end{description}

\enddeptgroup
