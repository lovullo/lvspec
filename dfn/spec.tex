% Specification Terminology
%
% Intended to be included within its own section.

\begindeptgroup{pm}
\label{s:specdfn}

\sigauth{NL}
Portions of this section conform to \rfc{2119}: ``Key words for use in RFCs to
Indicate Requirement Levels''. Certain terms from the RFC have been expressly
avoided; for example, the terms ``should'' and ``should not'' are not used
because this specification's audience has the capability to alter the
specification to resolve implementation issues and should exercise that ability.

\begin{description}
  \sigauth{NL}
  \dt{conforming implementation}
  An implementation that meets all of the requirements of this specification.

  \sigauth{NL}
  \dt{Deprecated}
  When used within context of this specification: the feature or requirement
  will be removed in future revisions of this specification.

  \sigauth{NL}
  \dt{Exempt}
  Denotes a condition under which an implementation need not conform to a
  requirement.

  \sigauth{NL}
  \dt{implementation}
  Software that implements this specification.

  \sigauth{NL}
  \dt{May; Optional}
  Alternatively, the adverb ``optionally''; denotes a requirement whose
  implementation is not required and may be omitted; such \shall be used only to
  provide flexibility for implementors to exercise their best judgment or to
  denote requirements that are not essential to the operation of the
  implementation.

  \sigauth{NL}
  \dt{Removed}
  A feature or requirement mentioned in an earlier revision of this
  specification has been removed and will not be superseded.

  \sigauth{NL}
  \dt{section}
  The section containing the mention of this term, as well as any subsections
  contained within it.

  \sigauth{NL}
  \dt{Shall; Must}
  Denotes a mandatory requirement. If an implementation does not implement such
  a requirement, then the implementation is not conforming and the results under
  such a requirement are \undefined.

  \sigauth{NL}
  \dt{Shall Not; Must Not}
  Denotes an absolute prohibition. If an implementation does not honor such a
  prohibition, then the implementation is not conforming and the results under
  such a requirement are \undefined.

  \sigauth{NL}
  \dt{Superseded}
  The feature or requirement mentioned in an earlier revision of this
  specification has been removed and replaced by another.

  \sigauth{NL}
  \dt{Undefined}
  The error condition results in behavior that is not defined by a particular
  standard; this term \shall be used only to document other standards or
  specifications---it \shall not be used to introduce undefined behavior into
  this specification.

  \sigauth{NL}
  \dt{Unspecified}
  The behavior is not determined by this specification or is inconsequential to
  the implementation's operation.

  \sigauth{NL}
  \dt{\S}
  Section reference

  \sigauth{NL}
  \dt{\P}
  Paragraph reference

  \sigauth{NL}
  \dt{$\leftarrow$}
  Store value into variable; as opposed to `$=$', which denotes a declarative
  (and consequently immutable) assignment.

  \sigauth{NL}
  \dt{$\square$}
  End of example or proof.

  \sigauth{NL}
  \dt{$\blacksquare$}
  End of formal definition.

  \sigauth{NL}
  \dt{$\lceil x\rceil$}
  The ceiling of $x$ (round up to the nearest integer).
\end{description}

\enddeptgroup
