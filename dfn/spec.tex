% Specification Terminology
%
% Intended to be included within its own section.

Portions of this subsection conform to \rfc{2119}: ``Key words for use in RFCs
to Indicate Requirement Levels''. Certain terms from the RFC have been expressly
avoided; for example, the terms ``should'' and ``should not'' are not used
because this specification's audience has the capability to alter the
specification to resolve implementation issues and should exercise that ability.

\begin{description}
  \dt{Shall; Must}
  Denotes a mandatory requirement.

  \dt{Shall Not; Must Not}
  Denotes an absolute prohibition.

  \dt{May; Optional}
  Alternatively, the adverb ``optionally''; denotes a requirement whose
  implementation is not required and may be omitted; such \shall be used only to
  provide flexibility for implementors to exercise their best judgment or to
  denote requirements that are not essential to the operation of the Software.

  \dt{Undefined}
  The error condition results in behavior that is not defined by a particular
  standard; this term \shall be used only to document other standards or
  specifications---it \shall not be used to introduce undefined behavior into
  this specification.

  \dt{Unspecified}
  The behavior is not determined by this specification or is inconsequential to
  the Software's operation.

  \dt{Deprecated}
  When used within context of this specification: the feature or requirement
  will be removed in future revisions of this specification.

  \dt{Superseded}
  The feature or requirement mentioned in an earlier revision of this
  specification has been removed and replaced by another.

  \dt{Removed}
  A feature or requirement mentioned in an earlier revision of this
  specification has been removed and will not be superseded.
\end{description}
